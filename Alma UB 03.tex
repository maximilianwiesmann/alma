\documentclass[a4paper, 11pt]{scrartcl}
\usepackage{graphicx}
\usepackage{tikz}
\usepackage[ngerman]{babel}
\usepackage{amsmath}
\usepackage{amsfonts}
\usepackage{amssymb}
\usepackage{amsthm}
\usepackage[utf8]{luainputenc}
\usepackage{graphics}
\usepackage[margin=10pt,font=small,labelfont=it]{caption}
\usepackage[left=3cm,right=3cm,top=2.5cm,bottom=3cm]{geometry}
\usepackage{graphicx}
\usepackage{csquotes}
\usepackage{cite}
\usepackage{hyperref}
\usepackage{url}
\usepackage{dsfont}
%\usepackage[onehalfspacing]{setspace}
\usepackage{qcircuit}
\usepackage{verbatim}
\usepackage{float}
\usepackage{mathtools}
\usepackage{xpatch}
\usepackage{lscape}
\usepackage{weva}
\usepackage{enumerate}
\usepackage{xfrac}
\usepackage{faktor}


\xyoption{color}

\newtheoremstyle{dotless}{}{}{}{}{\bfseries}{}{ }{}
\theoremstyle{dotless}

\newtheorem*{satz}{Satz}
\newtheorem{law}{Gesetz}
\newtheorem{postulat}{Postulat}
\newtheorem{axiom}{Axiom}
\newtheorem{defi}{Definition}
\newtheorem{kor}{Korollar}
\newtheorem*{lem}{Lemma}
\newtheorem{bez}{Bezeichnung}
\newtheorem{bem}{Bemerkung}
\newtheorem{bsp}{Beispiel}
\newtheorem{beh}{Behauptung}
\newtheorem*{exercise}{Übung}

\newcommand{\cgate}[1]{\color{red};*+<.6em>{#1}\POS ="i","i"+UR;"i"+UL **\dir{-};"i"+DL **\dir{-};"i"+DR **\dir{-};"i"+UR **\dir{-},"i" \qw}

\hypersetup{linkcolor=black,citecolor=blue,colorlinks=true}

\newcommand{\bra}[1]{{\left\langle{#1}\right\vert}}
\newcommand{\ket}[1]{{\left\vert{#1}\right\rangle}}
\newcommand{\floor}[1]{\left\lfloor #1\right\rfloor}
\newcommand{\ceil}[1]{\left\lceil #1\right\rceil}

\renewcommand\thesubsection{(\alph{subsection})}
\renewcommand{\autodot}{}

\title{Übungsblatt 3}
\author{Jonas Lippert \and Maximilian Wiesmann}
\subject{Algorithmische Mathematik I}
\subtitle{Übungsgruppe 4}
\date{}

\begin{document}
	\maketitle
	\section*{Aufgabe 1.}
	\subsection{}
	\minisec{Behauptung:}
	Die Collatz-Folge mit \texttt{n=n+1;} terminiert und liegt in $\Theta(\log n)$.
	\minisec{Beweis.}
	Es sei $(a_{k})_{n}$ die zu $n\in\mathbb{N}$ gehörige Collatz-Folge. Dabei bezeichne $|(a_{k})_{n}|$ ihre jeweilige Länge.
	Da sich die Folge im schnellsten Fall stets halbiert, folgt dann aus 
	\[
	(a_{k})_{n}=1\Leftrightarrow\frac{1}{2^{k}}=n\Leftrightarrow k=\log n,
	\]
	dass
	\[
	|(a_{k})_{n}|\in\Omega(\log n)\text{ mit }\alpha=1\text{ und }n_{0}=1.
	\]
	Wegen
	\[
	(a_{k})_{n}\leq2(a_{k+4})_{n}
	\]
	gilt außerdem 
	\[
	|(a_{k})_{n}|\leq4\cdot\log n
	\]
	und damit
	\[
	|(a_{k})_{n}|\in O(\log n)\text{ mit }\alpha=4\text{ und }n_{0}=1.
	\]
	\begin{flushright}
		$\Box$
	\end{flushright}
%
	\subsection{}
	\minisec{Behauptung:}
	Die Collatz-Folge mit \texttt{n=n+3;} terminiert genau dann, wenn $3\nmid n$.
	\minisec{Beweis.}
	\emph{Hilfslemma.} 
	\begin{equation}
		3\nmid n \Leftrightarrow 3\nmid\frac{n}{2}~\forall~n\in\mathbb{N}.
	\end{equation}
	\emph{Beweis.}
	\[
		3\mid\frac{n}{2}\Rightarrow\exists k\in\mathbb{N}:3k=\frac{n}{2}\Rightarrow 6k=n\Rightarrow 3\mid n.
	\]
	\[
		3\mid n\overset{n=2m}{\Rightarrow} 6\mid m  \Rightarrow\exists k\in\mathbb{N}:6k=m \Rightarrow 3\mid\frac{n}{2}.
	\]
	\emph{Hilfslemma.}
	\begin{equation}
		3\nmid n \Leftrightarrow 3\nmid n+3~\forall~n\in\mathbb{N}.
	\end{equation}
	\emph{Beweis.}
	\[
		3\mid(n+3)\Rightarrow\exists k\in\mathbb{N}:3k=n+3\Rightarrow3(k-1)=n\Rightarrow3\mid n.
	\]
	\[
		3\mid n \Rightarrow\exists k\in\mathbb{N}^{+}:3(k-1)=n\Rightarrow3k=n+3\Rightarrow3\mid(n+1).
	\]\\
	\emph{Beh.:} Für jeden Startwert $n$ gilt für die zugehörigen Folgeglieder $a(k)$ der Zusammenhang
	\begin{equation}
		3\nmid n \Leftrightarrow 3\nmid a(k)~\forall k.
	\end{equation}
	\emph{Beweis.}
	Es gelte $3\nmid a(k)$. Nach Definition von $a(k+1)$ folgt die Behauptung induktiv mit Hilfe von $(1)$, bzw. $(2)$.
	
	Die Collatz-Folgen terminieren mit $n\in\{1,2,4,5\}$ und bilden mit $n\in\{3,6\}$ eine Endlosschleife. Man sieht außerdem leicht durch Induktion, dass 
	\[
	\frac{n+3}{2}<n~\forall n\geq7 \Rightarrow a(k+2)<a(k)~\forall k.
	\]
	Zusammen mit $(3)$ folgt die Behauptung.
	\begin{flushright}
		$\Box$
	\end{flushright}
%
	\section*{Aufgabe 2.}
	\subsection*{(a)}
		Sei $n$ eine negative ganze Zahl und sei $z$ die $l$-stellige 2-Komplementdarstellung von $n$. Nach Lemma 2.6 gilt
		$$ z = K_2^l(-n) = \sum_{i=0}^{l-1}(2-1-z_i)2^i \quad\text{für}~-n-1=\sum_{i=0}^{l-1}z_i2^i $$ 
		Stehen nun $2l$ Ziffern zur Verfügung, so gilt
		$$-n-1=\sum_{i=0}^{2l-1}z_ib^i\quad \text{mit}~z_i=0~\text{für}~l\le i\le 2l-1$$
		$$\Rightarrow z = K_2^{2l}(-n) = \sum_{i=0}^{2l-1}(2-1-z_i)2^i = \sum_{i=0}^{l-1}(2-1-z_i)2^i + \sum_{i=l}^{2l-1}(2-1-0)2^i$$
		D.h. bei $2l$ Ziffern sind in der Komplementdarstellung die ersten $l$ Ziffern alle gleich 1 und die weiteren Ziffern genauso wie in $K_2^{l}(-n)$.
	\subsection*{(b)}
		Wie in (a) gilt
		$$ z = K_2^l(-n) = \sum_{i=0}^{l-1}(2-1-z_i)2^i \quad\text{für}~-n-1=\sum_{i=0}^{l-1}z_i2^i $$ 
		Die $i$-te Ziffer von $z$ ist also genau dann gleich 0, wenn $z_i=1$, und gleich 1, wenn $z_i=0$ ist. Vertauscht man nun also Nullen und Einsen miteinander, so erhält man an jeder Ziffer gerade $z_i$, also die $i$-te Ziffer der Binärdarstellung von $-n-1$, was nach Definition $-(z+1)$ entspricht.
	\section*{Aufgabe 3.}
	\subsection*{(a)}
		Ausmultiplizieren und Aufsummieren liefert:
		$$(19375573910)_{-10}=0\cdot (-10)^0+1\cdot (-10)^1+9\cdot (-10)^2 + 3\cdot (-10)^3+7\cdot (-10)^4+5\cdot(-10)^5$$$$+5\cdot (-10)^6+7\cdot (-10)^7+3\cdot (-10)^8+9\cdot (-10)^9 +1\cdot(-10)^{10}$$
		$$=0\cdot 10^0+9\cdot 10^2+7\cdot 10^4+5\cdot 10^6+3\cdot 10^8 +1\cdot10^{10}$$
		$$-(1\cdot 10^1+3\cdot 10^3+5\cdot 10^5+7\cdot 10^7+9\cdot 10^9)$$
		$$=(10305070900)_{10}-(9070503010)_{10}=(1234567890)_{10}$$
	\subsection*{(b)}
		Eine Darstellung von $(9230753)_{10}$ zur Basis $-10$ ist $(-11371367)_{-10}$. Wir überprüfen dies durch Nachrechnen:
		$$(-11371367)_{-10}=-(7\cdot (-10)^0+6\cdot (-10)^1+3\cdot (-10)^2+1\cdot (-10)^3$$$$+7\cdot (-10)^4+3\cdot (-10)^5+1\cdot (-10)^6+1\cdot (-10)^7)$$
		$$=-(7\cdot 10^0+3\cdot 10^2+7\cdot 10^4+1\cdot 10^6-(6\cdot 10^1+1\cdot 10^3+3\cdot 10^5+1\cdot 10^7))$$
		$$=(10301060)_{10}-(1070307)_{10}=(9230753)_{10}$$
	\subsection*{(c) \& (d)}
		Wir geben zwei Darstellungen einer beliebigen Zahl $x\in \mathbb{Z}$ zur Basis $-10$ an, von denen wir beweisen, dass sie i.A. verschieden sind, und zeigen so, dass stets eine solche Darstellung existiert und diese nicht immer eindeutig ist.\newline
		Sei $x=\pm\sum_{i=0}^{l-1}z_i10^i,~z_i\in \{0,\dots ,9\}$ die Darstellung von $x$ zur Basis 10. Für den Fall, dass $x$ negativ ist, müssen in den folgenden Darstellungen nur die Vorzeichen gewechselt werden, wir betrachten deshalb im Folgenden nur den Fall, dass $x$ positiv ist.\newline
		\underline{1. Darstellung:}\newline
		Fall 1: $l$ ist gerade.\newline
		Wir zeigen, dass die Darstellung
		$$x=\sum_{i=0}^{\floor{\frac{l}{2}}-1}z_{2i+1}^{\prime} 10^{2i+1}-\sum_{i=0}^{\ceil{\frac{l}{2}}-1}z_{2i}^{\prime} 10^{2i}=-\sum_{i=0}^{l-1}z_{i}^{\prime} (-10)^{i}$$
		mit
		$$z_j^{\prime}=\left\{\begin{array}{ll}
		10-z_j~,& \text{wenn}~j~ \text{gerade}\\
		z_j+1~,&\text{wenn}~j~ \text{ungerade}\\
		\end{array}\right.$$
		eine korrekte Darstellung von $x$ zur Basis $-10$ ist. Anmerkung: Ist $z_j=9$ und befinden wir uns im zweiten Fall der Darstellungsvorschrift, kommt es zu einem \glqq Übertrag\grqq. Dies stört uns aber nicht, wenn wir die $z_j$ formal als Koeffizienten in der Summe auffassen. Sollte der Fall eintreten, dass man z.B. als erste \glqq Ziffer\grqq $~-1$ erhält, kann man diese durch 19 ersetzen.\newline
		Fall 2: $l$ ist ungerade.\newline
		Wir wählen die Darstellung wie in Fall 1, nur ist dies in diesem Fall eine korrekte Darstellung von $x-10^l$ zur Basis $-10$. Es ist klar, dass wir dann auch eine Darstellung für $x$ erhalten, indem wir am Ende des Umwandlungsprozesses noch $10^l$ subtrahieren, d.h. eine 1 als erste Ziffer in der Darstellung zur Basis $-10$ ergänzen.\newline
		Beweis:\newline
		$$\sum_{i=0}^{\floor{\frac{l}{2}}-1}z_{2i+1}^{\prime} 10^{2i+1}-\sum_{i=0}^{\ceil{\frac{l}{2}}-1}z_{2i}^{\prime} 10^{2i}=\sum_{i=0}^{\floor{\frac{l}{2}}-1}(z_{2i+1}+1) 10^{2i+1}-\sum_{i=0}^{\ceil{\frac{l}{2}}-1}(10-z_{2i})10^{2i}$$
		$$=\sum_{i=0}^{\floor{\frac{l}{2}}-1}z_{2i+1} 10^{2i+1}+\sum_{i=0}^{\floor{\frac{l}{2}}-1} 10^{2i+1}-\sum_{i=0}^{\ceil{\frac{l}{2}}-1} 10^{2i+1}+\sum_{i=0}^{\ceil{\frac{l}{2}}-1}z_{2i} 10^{2i}$$
		$$=\left\{\begin{array}{ll}
		\sum_{i=0}^{l-1}z_i10^i-10^{l}=x-10^{l}\quad&\text{für ungerades }l\\
		\sum_{i=0}^{l-1}z_i10^i=x\quad&\text{für gerades }l\\
		\end{array}\right.
		$$
		\underline{2. Darstellung:}\newline
		Wir zeigen, dass die Darstellung
		$$x=\sum_{i=0}^{\ceil{\frac{l}{2}}-1}z_{2i}^{\prime} 10^{2i}-\sum_{i=0}^{\floor{\frac{l}{2}}-1}z_{2i+1}^{\prime} 10^{2i+1}=\sum_{i=0}^{l-1}z_{i}^{\prime} (-10)^{i}$$
		mit
		$$z_j^{\prime}=\left\{\begin{array}{ll}
		z_j+1~,& \text{wenn}~j~ \text{gerade}\\
		10-z_j~,&\text{wenn}~j~ \text{ungerade}\\
		\end{array}\right.$$
		eine korrekte Darstellung von $x+1$ zur Basis $-10$ ist, wenn $l$ ungerade ist, und eine korrekte Darstellung von $x+1-10^l$, wenn $l$ gerade ist (analog zu oben; entsprechend muss dann von der $-10$-adischen Darstellung noch 1 subtrahiert respektive 1 subtrahiert und $10^l$ addiert werden).\newline
		Beweis:
		$$\sum_{i=0}^{\ceil{\frac{l}{2}}-1}z_{2i}^{\prime} 10^{2i}-\sum_{i=0}^{\floor{\frac{l}{2}}-1}z_{2i+1}^{\prime} 10^{2i+1}=\sum_{i=0}^{\ceil{\frac{l}{2}}-1}(z_{2i}+1) 10^{2i}-\sum_{i=0}^{\floor{\frac{l}{2}}-1}(10-z_{2i+1}) 10^{2i+1}$$
		$$=\sum_{i=0}^{\ceil{\frac{l}{2}}-1}z_{2i} 10^{2i} + \sum_{i=0}^{\ceil{\frac{l}{2}}-1} 10^{2i} - \sum_{i=0}^{\floor{\frac{l}{2}}-1} 10^{2i+2} + \sum_{i=0}^{\floor{\frac{l}{2}}-1}z_{2i+1} 10^{2i+1}$$
		$$=\sum_{i=0}^{l-1}z_i10^i + \sum_{i=0}^{\ceil{\frac{l}{2}}-1} 10^{2i} - 
		\sum_{i=1}^{\floor{\frac{l}{2}}} 10^{2i}$$
		$$=\left\{\begin{array}{ll}
		\sum_{i=0}^{l-1}z_i10^i+10^0=x+1\quad&\text{für ungerades }l\\
		\sum_{i=0}^{l-1}z_i10^i +10^0 - 10^l=x+1-10^l\quad&\text{für gerades }l\\
		\end{array}\right.
		$$
		Diese beiden Darstellungen sind i.A. nicht gleich. Um dies zu zeigen, betrachten wir das Beispiel aus Teil (b):
		$$(9230753)_{10}\overset{\text{1. Darst.}}{=}(-11371367)_{-10}\overset{\text{2. Darst.}}{=}(190830854)_{-10}$$
		Folglich ist die Darstellung einer Zahl zur Basis $-10$ nicht immer eindeutig, solche Darstellungen existieren aber.\qed
\end{document}